\documentclass[12pt,letterpaper]{article}
\usepackage[utf8]{inputenc}
\usepackage[spanish, english]{babel}
\usepackage{graphicx}
\usepackage{lettrine}
\usepackage{enumitem}
\usepackage[left=3cm,right=3cm,top=3cm,bottom=3cm]{geometry}
\usepackage{float} 
\usepackage{amsmath}
\usepackage{stackrel} 
\usepackage{multirow}
\usepackage{enumerate}
\renewcommand{\labelitemi}{$-$}
\renewcommand{\labelitemii}{$\cdot$}

\providecommand{\keywords}[1]
{
  \small	
  \textbf{\textit{Keywords: }} #1
}

\providecommand{\pclave}[1]
{
  \small	
  \textbf{\textit{Palabras Clave:}} #1
}
\begin{document}

\title{Caratula}
\begin{titlepage}
\begin{figure}[htb]
\begin{center}
\includegraphics[width=4cm]{./Imagenes/logo.png}
\end{center}
\end{figure}
\vspace*{-0.25in}
\begin{center}
\large{UNIVERSIDAD PRIVADA DE TACNA}\\
\vspace*{-0.025in}
INGENIERIA DE SISTEMAS  \\

\vspace*{0.5in}
\begin{large}
TITULO:\\
\end{large}

\vspace*{0.1in}
\begin{Large}

\textbf{"Frameworks de pruebas"} \\
\end{Large}

\vspace*{0.3in}
\begin{Large}
\textbf{CURSO:} \\
\end{Large}

\vspace*{0.1in}
\begin{large}
CALIDAD Y PRUEBAS DE SOFTWARE\\
\end{large}

\vspace*{0.3in}
\begin{Large}
\textbf{DOCENTE:} \\
\end{Large}

\vspace*{0.1in}
\begin{large}
 Ing. Patrick Cuadros Quiroga\\
\end{large}

\vspace*{0.2in}
\vspace*{0.1in}
\begin{large}
Integrantes:\\
\begin{flushleft}
Maldonado Cancapi, Carlos Alejandro\hfill(2018000660) \\
Villanueva Yucra, Josue Joel\hfill(2018000722)\\
Contreras Murguia, Jose Manuel \hfill(2016056346)\\
Rojas Bedregal, Brian Erik\hfill(2018060904)\\
Mamani Laura, Juan Carlos \hfill(2017059565)\\

\end{flushleft}
\end{large}

\vspace*{0.1in}
\begin{large}
Tacna - Perú\\
2021\\

\end{large}
\end{center}

\end{titlepage}

\selectlanguage{spanish}
\begin{abstract}
    Por medio de la utilización de este framework se podrá tener una amplia visión de todo lo que el testing de software implica. Como se plantea en el transcurso de este documento, la idea central es tener una amplia gama de conocimientos que permitan el aprendizaje y la realización de los procesos involucrados en el testing de software.

    Esta informacion sirve como base de conocimientos para la implementación de un proceso de testing que asegure la calidad de los productos de software. El centro del proyecto es el framework, que propone el proceso de testing desde el punto de vista teórico y práctico y además funciona como un repositorio de conocimientos para permitir a quienes estén interesados en este proceso acceder a información concisa para ser llevada a la realidad de los proyectos que incluyan testing de software, tanto en el ámbito empresarial como académico.

\end{abstract}
\pclave{estructura de clases, patrón de diseño de software.}

\begin{center}\rule{1\textwidth}{0.05mm} \end{center}

\selectlanguage{english}
\begin{abstract}
    Through the use of this framework, you can have a broad vision of everything that software testing involves. As stated in the course of this document, the central idea is to have a wide range of knowledge that allows learning and carrying out the processes involved in software testing.

    This information serves as a knowledge base for the implementation of a testing process that ensures the quality of software products. The center of the project is the framework, which proposes the testing process from a theoretical and practical point of view and also functions as a repository of knowledge to allow those interested in this process to access concise information to be taken to the reality of projects that include software testing, both in business and academia.

\end{abstract}
\keywords{class structure, software design pattern.}

\selectlanguage{spanish}


\section{Introducción}
En la actualidad, el testing ha adquirido mayor relevancia 
en las organizaciones en lo que al desarrollo de software se refiere, ya que se ha hecho evidente su importancia por el ahorro que representa detectar tempranamente los errores del software.
La complejidad propia del software hace que este sea susceptible 
de tener errores y un software que no pase por un proceso de testing
puede causar importantes pérdidas a nivel económico, daños ambientales 
e incluso puede costar la vida de las personas; puede resultar en un software
no usable (en términos de usabilidad o con múltiples defectos), teniendo como 
resultado procesos inestables o con resultados inútiles. Sin embargo, a pesar 
de la importancia del testing en la calidad del software, y de los beneficios 
que tiene, no existe en la actualidad un framework que reúna los procesos, las 
actividades, los roles involucrados, los formatos requeridos, etc.

\section{Desarrollo}

\subsection{UFT One}
\begin{figure}[h]
    \begin{center}
    \includegraphics[width=10cm]{./Imagenes/image1.png}
    \caption{UFT One.}
    \label{rg2}
    \end{center}
    \end{figure}
UFT One (anteriormente conocido como UFT) es una popular herramienta comercial para probar aplicaciones web, de escritorio, móviles y RPA. Se ha ampliado para incluir un buen conjunto de capacidades para las pruebas de API. Al admitir múltiples plataformas para la aplicación de destino bajo prueba (AUT), UFT One proporciona una opción conveniente para probar el AUT que opera en computadoras de escritorio, web y dispositivos móviles.
Interfaz de usuario intuitiva para crear, ejecutar e informar pruebas de API
Soporte para generar pruebas API a partir de documentos WADL
Las acciones, actividades y parámetros de las pruebas se pueden visualizar en diagramas

\subsection{Estudio Katalon}

\begin{figure}[h]
    \begin{center}
    \includegraphics[width=10cm]{./Imagenes/image2.png}
    \caption{Estudio Katalon}
    \label{rg3}
    \end{center}
    \end{figure}
Katalon Studio es una potente y completa solución de automatización para probar API, Web, 
aplicaciones móviles y de escritorio. También tiene un amplio conjunto de funciones para este 
tipo de pruebas y es compatible con múltiples plataformas, incluidas Windows, macOS y Linux.
Aprovechando los motores Selenium y Appium, Katalon Studio proporciona un entorno integrado de 
forma única para los probadores que encuentran dificultades para integrar e implementar diferentes 
marcos y bibliotecas para usar Selenium y Appium, así como para aquellos que ya están familiarizados 
con estos motores.
Admite SOAP y RESTful para pruebas de API y servicios
Cientos de palabras clave integradas para crear casos de prueba
Admite BDD Cucumber para expresar el escenario de prueba en lenguajes naturales
Se puede utilizar tanto para pruebas automatizadas como exploratorias



\subsection{SoapUI}
\begin{figure}[h]
    \begin{center}
    \includegraphics[width=10cm]{./Imagenes/image3.png}
    \caption{SoapUI}
    \label{rg4}
    \end{center}
    \end{figure}
SoapUI no es una herramienta de automatización de pruebas para pruebas web 
o de aplicaciones móviles, pero puede ser una herramienta de elección para probar 
API y servicios. Es una herramienta de prueba funcional sin cabeza diseñada 
específicamente para pruebas API.
SoapUI admite servicios REST y SOAP. Los probadores de automatización de API pueden 
usar la versión de código abierto o pro. La edición pro tiene una interfaz fácil de 
usar y varias características avanzadas como asistente de afirmación, editor de formularios 
y generador de consultas SQL. SoapUI es una herramienta de la suite ReadyAPI, ofrecida por SmartBear.

\begin{itemize}
    \item	Generar pruebas fácilmente usando arrastrar y soltar, apuntar y hacer clic
    \item	Potentes pruebas basadas en datos con datos de archivos y bases de datos
    \item	Prueba asincrónica

\end{itemize}




\subsection{Tricentis Tosca }
\begin{figure}[h]
    \begin{center}
    \includegraphics[width=10cm]{./Imagenes/image4.png}
    \caption{Tricentis Tosca}
    \label{rg5}
    \end{center}
    \end{figure}
Hay algunas plataformas de pruebas continuas que proporcionan conjuntos de herramientas integrales para respaldar la mayoría, si no todas, las actividades de prueba que van desde el diseño de pruebas y la automatización de pruebas hasta informes de pruebas y análisis. Tricentis Tosca es uno de ellos.
Esta herramienta tiene muchas características, como paneles, análisis, integraciones y ejecuciones distribuidas para respaldar la integración continua y las prácticas de DevOps. Además, ofrece una interfaz de usuario amigable y un rico conjunto de características para diseñar, implementar, ejecutar, administrar y optimizar pruebas API.


\begin{itemize}
    \item Se puede integrar fácilmente para ser una parte crucial de los procesos de DevOps.
    \item Las pruebas de API se pueden realizar en navegadores, dispositivos móviles y plataformas.
    \item Se habilitan varios protocolos y estándares, incluidos HTTP (s) JMS, AMQP, Rabbit MQ, TIBCO EMS, SOAP, REST e IBM MQ
\end{itemize}



\subsection{TestComplete}
\begin{figure}[h]
    \begin{center}
    \includegraphics[width=10cm]{./Imagenes/image5.png}
    \caption{TestComplete}
    \label{rg5}
    \end{center}
    \end{figure}
TestComplete sigue estando en la lista este año por su potente y completo conjunto de funciones para las pruebas de aplicaciones web, móviles y de escritorio. Los evaluadores pueden utilizar JavaScript, VBScript, Python o C ++ Script para escribir scripts de prueba.
Al igual que UFT One, TestComplete tiene un motor de reconocimiento de objetos que puede detectar con precisión los elementos dinámicos de la interfaz de usuario. Este motor es especialmente útil en aplicaciones que tienen interfaces de usuario dinámicas y que cambian con frecuencia.

\subsection{IBM Rational Functional Tester (RFT)}
\begin{figure}[h]
    \begin{center}
    \includegraphics[width=10cm]{./Imagenes/image6.png}
    \caption{IBM Rational Functional Tester (RFT)}
    \label{rg6}
    \end{center}
    \end{figure}
IBM Rational Functional Tester es una herramienta de 
automatización de pruebas diseñada para probar aplicaciones
 que se desarrollan utilizando diferentes lenguajes y tecnologías
  como Web, .Net, Java, Visual Basic, Siebel, SAP, PowerBuilder, Adobe
   Flex y Dojo Toolkit. También es una plataforma de prueba basada en 
   datos para pruebas funcionales y de regresión.
\begin{itemize}
    \item Tecnología avanzada ScriptAssure
    \item Detección de datos más temprana
    \item Las secuencias de comandos de prueba permiten a los usuarios elegir entre Java o Visual Basic .NET

\end{itemize}
\subsection{Ranorex}
\begin{figure}[h]
    \begin{center}
    \includegraphics[width=10cm]{./Imagenes/image7.png}
    \caption{Ranorex}
    \label{rg7}
    \end{center}
    \end{figure}
Ranorex, que existe desde hace muchos años, ofrece un conjunto completo y profesional de funciones para pruebas de API, escritorio, dispositivos móviles y Web. Aprovechando su experiencia en la automatización de pruebas basada en escritorio, Ranorex tiene capacidades avanzadas para la identificación, edición y administración de elementos de la interfaz de usuario.
Al igual que Katalon Studio, Ranorex facilita las pruebas de automatización para los probadores con su GUI amigable e intuitiva, grabación / reproducción y generación de scripts.
\subsection{Telerik Test Studio}
\begin{figure}[h]
    \begin{center}
    \includegraphics[width=10cm]{./Imagenes/image8.png}
    \caption{Telerik Test Studio}
    \label{rg8}
    \end{center}
    \end{figure}
Desarrollado como una herramienta de prueba de software 
basada en Windows, Telerik Test Studio es ampliamente conocido 
por pruebas funcionales web y de escritorio, pruebas de rendimiento 
de software y pruebas de aplicaciones móviles. Esta solución permite 
tanto capacidades basadas en código como un enfoque sin código, lo que 
garantiza una calidad óptima de la aplicación con los resultados más destacados.
\begin{itemize}
    \item Grabador intuitivo compatible con varios navegadores
    \item Los resultados de las pruebas se muestran en el panel ejecutivo
    \item Detección inteligente de elementos híbridos
    \item Programación y ejecuciones simultáneas
    \item Gestión de elementos de prueba

\end{itemize}
\section{Conclusiones}
Actualmente el testing de frameworks puede apoyarse en una creciente cantidad de metodologías y aplicaciones diseñadas exclusivamente para la etapa de pruebas, incluyendo la gestión del proceso de software testing, la administración y seguimiento de defectos, la administración de los casos de prueba, la automatización de pruebas.
El testing debe estar integrado en todo el proceso de desarrollo para poder aportar a la calidad del producto. La formalidad con la cual sea realizado es fundamental para entregar a los clientes productos con excelente calidad, así, el proceso de desarrollo y el de pruebas deben estar soportados por una metodología formal para ejecutar el proyecto.

\section{Recomendaciones}
Un punto importante para el éxito de las pruebas de un producto software, es permitirle al equipo de pruebas involucrarse desde el inicio del proyecto, para que entiendan el negocio del cliente, comprendan la importancia y criticidad del producto que van a verificar y finalmente, durante la ejecución, puedan enfocarse en aquellos aspectos más relevantes para el usuario final, desde el punto de vista de la funcionalidad, la usabilidad, el desempeño.
Los temas presentados a lo largo de este documento, permiten identificar los elementos críticos del testing. Esto facilita el proceso de gestión del mismo ya que permite diferenciar los aspectos relevantes y da pautas de cómo poner en práctica los conceptos vistos.

	

\begin{thebibliography}{XXX0000}

\bibitem - Cuervo, (21 de junio de 2011). Evaluación y análisis de rendimiento de los frameworks de persistencia Hibernate y Eclipselink. Obtenido de revistas sum: https://revistasum.umanizales.edu.co/ojs/index.php/ventanainformatica/article/view/155
\bibitem - Espinosa, S. G. (2016 de 03 de 1). Personalización del proceso de pruebas unitarias empleando la herramienta NUnit. Obtenido de serie cientifica: https://publicaciones.uci.cu/index.php/serie/article/view/811
\bibitem - Ríos, J. R. (23 de 09 de 2016). Evaluación de los Frameworks en el Desarrollo de Aplicaciones Web con Python. Obtenido de revistas unla: http://revistas.unla.edu.ar/software/article/view/1149
\bibitem - sierra, f. (01 de marzo de 2017). Estudio y análisis de los framework en php basados en el modelo vista controlador para el desarrollo de software orientado a la web. Obtenido de Investigación y desarrollo en TIC: http://revistas.unisimon.edu.co/index.php/identic/article/view/2480
\bibitem - sierra, f. (01 de marzo de 2017). Estudio y análisis de los framework en php basados en el modelo vista controlador para el desarrollo de software orientado a la web. Obtenido de Investigación y desarrollo en TIC: http://revistas.unisimon.edu.co/index.php/identic/article/view/2480 
\bibitem - Paloma Díaz, Susana Montero, Ignacio Aedo. (2005). Ingeniería de la web y patrones de diseño.
\bibitem - Villón Moreno, M. S. (abril de 2019). Plataforma tecnológica para contribuir a la planeación urbana en la ciudad de Guayaquil dirigido a la transportación enfocado al diseño de pruebas que permitan la verificación y validación del software usando herramientas de pruebas ágiles. Obtenido de repositorio institucional de la universidad de guayaquil: http://repositorio.ug.edu.ec/handle/redug/39670
\bibitem -  Paloma D´ıaz, Susana Montero, Ignacio Aedo. (2005). Ingenier´ıa de la web
\end{thebibliography}
\end{document}